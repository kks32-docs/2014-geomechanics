\documentclass[11pt,a4]{article}
\usepackage{a4wide}

%*********************************** To copy ligatures ********************************************************************************************* %
\usepackage[ansinew]{inputenc}
\usepackage[T1]{fontenc}
\usepackage{libertine}

\pdfglyphtounicode{f_f}{FB00}
\pdfglyphtounicode{f_f_i}{FB03}
\pdfglyphtounicode{f_f_l}{FB04}
\pdfglyphtounicode{f_i}{FB01}

\pdfgentounicode=1

%**************************************************************************************************************************************************** %
\usepackage{url}
%\usepackage{hyperref}
\usepackage[pdftex, plainpages=false, pdfpagelabels, pdfpagelayout=useoutlines, bookmarks, bookmarksopen=true, bookmarksnumbered=true, breaklinks=true, linktocpage, pagebackref=false, colorlinks=true, linkcolor=blue, urlcolor=blue, citecolor=blue, anchorcolor=green, hyperindex=false, hyperfigures]{hyperref}
\def\url@leostyle{%
  \@ifundefined{selectfont}{\def\UrlFont{\sf}}{\def\UrlFont{\small\ttfamily}}}
%\usepackage{leo}
\usepackage{authblk}

\title{Underwater Granular Flows down Inclined Planes}
\author[1]{Krishna Kumar}
\author[1]{Kenichi Soga}
\author[2]{Jean-Yves Delenne}

\affil[1]{Department of Engineering, University of Cambridge, Cambridge, CB2 1PZ, UK}
\affil[2]{IATE, UMR1208 INRA-CIRAD-Montpellier Supagro-UM2, 2 place Pierre Viala, 34060 Montpellier cedex 01, France}

\renewcommand\Authands{ and }
\date{}

\begin{document}
\maketitle

Geophysical hazards, such as avalanches, debris flows and submarine landslides, involve rapid mass movement of granular solids, water and air as a single-phase system. The momentum transfer between the discrete and the continuous phases significantly affects the dynamics of the flow. The initiation and propagation of submarine granular flows depend mainly on the slope, density, and quantity of the material destabilised. Although certain macroscopic models are able to capture simple mechanical behaviours, the complex physical mechanisms that occur at the grain scale, such as hydrodynamic instabilities, the formation of clusters, collapse, and transport, have largely been ignored. In order to describe the mechanism of saturated and/or immersed granular flows, it is important to consider both the dynamics of the solid phase and the role of the ambient fluid. In particular, when the solid phase reaches a high volume fraction, it is important to consider the strong heterogeneity arising from the contact forces between the grains, the drag interactions which counteract the movement of the grains, and the hydrodynamic forces that reduce the weight of the solids inducing a transition from dense compacted to a dense suspended flow. Hence, it is important to understand the mechanism of underwater granular flows at the grain scale. A pending research issue is the parameterisation of interactions between the water phase and the sediment phase. Owing to the number of flow variables involved and measurement imprecision, estimating such parameters from laboratory experiments remains difficult.

In this study, two-dimensional sub-grain scale numerical simulations are performed to understand the local rheology of dense granular flows in fluid. The Discrete Element (DEM) technique is coupled with the Lattice Boltzmann Method (LBM), for fluid-grain interactions, to understand the evolution of immersed granular flows. The fluid phase is simulated using Multiple-Relaxation-Time LBM (LBM-MRT) for numerical stability. The Eulerian nature of the LBM formulation, together with the common explicit time step scheme of both LBM and DEM makes this coupling strategy an efficient numerical procedure for systems dominated by both grain--fluid and grain--grain interactions. In order to simulate interconnected pore space in 2D, a reduction in the radius of the grains (hydrodynamic radius) is assumed during LBM computations. By varying the hydrodynamic radius of the grains, granular materials of different permeabilities can be simulated. A parametric analysis is performed to assess the influence of the granular characteristics (initial packing, permeability, slope of the inclined plane) on the evolution of flow and run-out distances. The effect of hydrodynamic forces and hydroplaning on the run-out evolution is analysed by comparing the mechanism of energy dissipation and flow evolution in dry and immersed granular flows. Voronoi tesselation was used to study the evolution of local density and water entrainment at the flow front. 


\end{document}
